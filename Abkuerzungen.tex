% !TEX root = Projektdokumentation.tex

% Es werden nur die Abkürzungen aufgelistet, die mit \ac definiert und auch benutzt wurden. 
%
% \acro{VERSIS}{Versicherungsinformationssystem\acroextra{ (Bestandsführungssystem)}}
% Ergibt in der Liste: VERSIS Versicherungsinformationssystem (Bestandsführungssystem)
% Im Text aber: \ac{VERSIS} -> Versicherungsinformationssystem (VERSIS)

% Hinweis: allgemein bekannte Abkürzungen wie z.B. bzw. u.a. müssen nicht ins Abkürzungsverzeichnis aufgenommen werden
% Hinweis: allgemein bekannte IT-Begriffe wie Datenbank oder Programmiersprache müssen nicht erläutert werden,
%          aber ggfs. Fachbegriffe aus der Domäne des Prüflings (z.B. Versicherung)

% Die Option (in den eckigen Klammern) enthält das längste Label oder
% einen Platzhalter der die Breite der linken Spalte bestimmt.
\begin{acronym}[WWWWW]
	\acro{BBW}{\textsc{Bugenhagen} Berufsbildungswerk}
	\acro{BvB}{Berufsvorbereitende Bildungsmaßnahme}
	\acro{DMS}{Dokumentenmanagementsystem}
	\acro{ITSC}{IT-Service-Center}
	\acro{AD}{Active Directory}
	\acro{OP}{OpenProject}
	\acro{BS}{Betriebssystem}
	\acro{COLD}{Computer Output on Laserdisk}
	\acro{BBWTS}{Bugenhagen Ticketsystem}
	\acro{SMB}{Server Message Block}
	\acro{FTP}{File Transfer Protocoll}
	\acro{IMAP}{ Internet Message Access Protocol}
	\acro{POP}{Post Office Protocol Version 3}
	\acro{SMTP}{Simple Mail Transfer Protocol}
	\acro{OS}{Open Source}
	\acro{ESX}{VMware ESXi}
	\acro{LTS}{long-term support}
	\acro{WIN10P}{Windows 10 Pro}
	\acro{MFP}{Multi Function Printer}
	\acro{OCR}{Optical Character Recognition}
	\acro{ICR}{Intelligent Character Recognition}
	\acro{NCI}{Non Coded Information}
	\acro{CI}{Coded Information}
	\acro{RZ}{Rechenzentrum}
	\acro{TCO}{Total Cost of Ownership}
	\acro{BSI}{Bundesamt für Sicherheit in der Informationstechnik}
	\acro{SSO}{single Sign-on}
\end{acronym}
